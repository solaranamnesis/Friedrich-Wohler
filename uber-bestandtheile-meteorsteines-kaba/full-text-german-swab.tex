\documentclass[a4paper, 11pt, oneside]{article}
\usepackage[utf8]{inputenc}
\usepackage[T1]{fontenc}
\usepackage[ngerman]{babel}
\usepackage{fbb} %Derived from Cardo, provides a Bembo-like font family in otf and pfb format plus LaTeX font support files
\usepackage{booktabs}
\setlength{\emergencystretch}{15pt}
\usepackage{fancyhdr}
\usepackage{graphicx}
\graphicspath{ {./} }
\usepackage{microtype}
\usepackage{yfonts}
\usepackage[figurename=]{caption}
\usepackage[titles]{tocloft}
\usepackage{sectsty}
\sectionfont{\Huge}
\subsectionfont{\LARGE}
\subsubsectionfont{\LARGE}
\begin{document}
\swabfamily
\renewcommand{\contentsname}{
\swabfamily{Inhaltsverzeichnis}
}
\begin{titlepage} % Suppresses headers and footers on the title page
	\centering % Centre everything on the title page
	%\scshape % Use small caps for all text on the title page

	%------------------------------------------------
	%	Title
	%------------------------------------------------
	
	\rule{\textwidth}{1.6pt}\vspace*{-\baselineskip}\vspace*{2pt} % Thick horizontal rule
	\rule{\textwidth}{0.4pt} % Thin horizontal rule
	
	\vspace{1\baselineskip} % Whitespace above the title
	
	{\scshape\Huge Sitzungsberichte\\[1.25pt] der Kaiserlichen\\[1.25pt] Akademie der Wissenschaften.\\[1.25pt] mathematisch-naturwissenschaftliche\\[1.25pt] klasse.\\[1.25pt]}
	
	\vspace{1\baselineskip} % Whitespace above the title

	\rule{\textwidth}{0.4pt}\vspace*{-\baselineskip}\vspace{3.2pt} % Thin horizontal rule
	\rule{\textwidth}{1.6pt} % Thick horizontal rule
	
	\vspace{1\baselineskip} % Whitespace after the title block
	
	%------------------------------------------------
	%	Subtitle
	%------------------------------------------------
	
	{\LARGE Sitzung vom 11. November 1858.} % Subtitle or further description
	
	\vspace*{1\baselineskip} % Whitespace under the subtitle
	
    {\LARGE Von dem c. M. Prof. F. W"ohler in G"ottingen.} % Subtitle or further description
    
	%------------------------------------------------
	%	Editor(s)
	%------------------------------------------------
    \vspace*{\fill}

	\vspace{1\baselineskip}

	{\Large Wien 1859.}
	
	{\Large\scshape{Aus der K. K. Hof- und Staatsdruckerei.}}
	
	\vspace{0.5\baselineskip} % Whitespace after the title block

    \Large\scshape Internet Archive Online Edition  % Publication year
	
	{\Large Namensnennung Nicht-kommerziell Weitergabe unter gleichen Bedingungen 4.0 International} % Publisher
\end{titlepage}
\setlength{\parskip}{1mm plus1mm minus1mm}
\clearpage
\tableofcontents
\clearpage
\LARGE
\pagestyle{fancy}
\fancyhf{}
\cfoot{\swabfamily{\thepage}}
\section{\swabfamily{Sitzung vom 11. November 1858. --- Über die Bestandteile des Meteorsteines von Kaba in Ungarn.}}
\paragraph{}
Die Fragmente von dem am 15. April 1857 bei Kaba in Ungarn gefallenen Meteoriten, die mir zur Analyse dienten, verdanke ich der G"ute des Vorstandes des k. k. Hof-Mineralien-Kabinettes zu Wien, Herrn Dr. H"ornes, der "uber das Ph"anomen des Falles und die "au"sere Beschaffenheit des Steines im XXXI. Bande der Sitzungsberichte der mathematisch-naturwissenschaftlichen Classe der kaiserl. Akademie eine n"ahere Beschreibung mitgeteilt hat. Seinem Wunsche, die Analyse vorzunehmen, entsprach ich umso lieber, als das in der Tat ganz ungew"ohnliche Aussehen dieses Steines auf eine ungew"ohnliche Zusammensetzung schlie"sen lie"s.

Die mir "ubergebenen kleinen Fragmente waren ohne Rinde, hatten eine dunkelgraue Farbe und einen erdigen Bruch und waren leicht zerbrechlich und zerreiblich. In der erdigen grauen Grundmasse war hier und da ein wei"ses, und ein gr"unliches, ganz wie Olivin aussehendes Mineral zu bemerken. Die auch schon in mehreren anderen Meteorsteinen beobachteten sonderbaren leicht ausl"osbaren schwarzen K"ugelchen waren in diesem Stein in ungew"ohnlich gro"ser Anzahl enthalten. Sie waren sehr spr"ode, zeigten nach dem Zerdr"ucken unter die Mikroskope im Innern einen leeren Raum und bestanden aus einem farblosen, sehr kristallinischen, und einem schwarzen Mineral. Die kleine, zu Gebote stehende Menge gestattete nicht, eine besondere Analyse davon zu machen. Von metallischen Teilchen war in diesen Fragmenten keine Spur zu entdecken; dennoch lenkten sie schwach die Magnetnadel ab, und aus dem Pulver lie"sen sich vermittelst des Magnetes sehr kleine Teilchen von metallischen Eisen ausziehen. Aus der oben erw"ahnten Beschreibung des ganzen Steines, wonach er auf der einen Seite viele gl"anzende Metallk"orner enth"alt, ist daher zu schlie"sen, dass er sehr ungleich gemengt sein muss. Das folgende analytische Resultat bezieht sich also nur auf den erdigen dunkelgrauen Theil dieses Steines. In 100 Theilen desselben wurden gefunden:
\begin{center}
    \begin{tabular}{ l r } 
    Kohle & 0,58.\\
    Eisen & 2,88.\\
    Nickel & 1,37.\\
    Kupfer & 0,01.\\
    Chromeisenstein & 0,89.\\
    Magnetkies & 3,55.\\
    Eisenoxydul & 26,20.\\
    Magnesia & 22,39.\\
    Thonerde & 5,38.\\
    Kalk & 0,66.\\
    Kalt (und Natron?) & 0,30.\\
    Manganoxydul & 0,05.\\
    Kiesels"aure & 34,24.\\
    Kobalt & in unbestimmbarer Menge.\\
    Phosphor & in unbestimmbarer Menge.\\
    Unbekannte Materie & in unbestimmbarer Menge.\\
     & 98,50.\\
    \end{tabular}
\end{center}
\paragraph{}
Dieser Stein enth"alt also die gew"ohnlichen Bestandteile der nicht metallischen Meteoriten, er ist ein Gemenge von einem durch Salzs"aure leicht zersetzbaren Magnesia-Eisenoxydul-Silikat und von Silikaten, die durch diese S"aure nicht zersetzt werden; er enth"alt au"serdem kobalt- und phosphorhaltiges Nickeleisen, Schwefeleisen, Chromeisenstein und als ungew"ohnlichen Bestandteil schwarze, amorphe Kohle. Was die unbekannte Materie betrifft, so will ich weiter unten noch einige Worte dar"uber sagen.

Es wurden zwei Analysen von den Steinen gemacht, die eine mit 2,827 Grm. durch Aufschlie"sung mit kohlensaurem Kali-Natron, die andere mit 3,008 Grm. durch Flusss"aure. Die erstere gab den obigen Kiesels"auregehalt, die andere, bei welcher der Verlust als Kiesels"auregehalt genommen werden musste, gab gerade 1 Prozent mehr.

Der Gehalt an metallischem Eisen konnte nicht direkt bestimmt werden, sondern wurde nach der Menge des Nickels berechnet, mit der Annahme, dass der Stein das den Meteoriten gew"ohnliche, in Salzs"aure schwer l"osliche Nickeleisen enthalte. Denn er entwickelt mit dieser S"aure kaum sichtbare Spuren von Wasserstoffgas. Die S"aure l"ost aber selbst in der K"alte viel auf und diese Aufl"osung enth"alt dann viel Magnesia und Eisenoxydul. Es wurde daher der "ubrige Eisengehalt, mit Ausnahme des an Schwefel gebundenen, als Oxydul in Rechnung gebracht. Es wurden im Ganzen 34,57 Prozent Eisenoxyd erhalten, der gefundene Schwefelgehalt betrug 1,42 Prozent, entsprechend 3,55 Magnetkies. Dass der Stein diesen und nicht einfach Schwefeleisen enthalte, wird daraus wahrscheinlich, dass er mit Salzs"aure erst in der W"arme deutlich Schwefelwasserstoff entwickelt und dass der R"uckstand dann freien Schwefel enth"alt. --- Kobalt und Phosphor waren mit Sicherheit nachzuweisen, ihre Mengen w"aren aber nur mit Anwendung von mehr Material zu bestimmen gewesen.

Der Kohlegehalt verr"at sieh zun"achst dadurch, dass der schwarze R"uckstand von der Aufl"osung des Steines in Salzs"aure selbst nach langem Kochen mit K"onigswasser schwarz blieb, dass er sich aber nach dem Ausw"uschen und Trocknen an der Luft rasch zimmtbraun brennen lie"s, eine Eigenschaft, die auch der nicht mit S"aure behandelte Stein hat. Zur quantitativen Bestimmung des Kohlenstoffes wurde eine abgewogene Menge des fein zerriebenen Steines in einem langsamen Strom von durch Kalihydrat sorgf"altig gereinigtem Sauerstoffgas zum Gl"uhen erhitzt, das Gas dann, zur Entfernung der gleichzeitig sich bildenden schwefligen S"aure, durch ein Rohr mit Bleisuperoxyd und von da durch einen gewogenen Kaliapparat geleitet. Dieser bestand aus dem Liebig’schen Kugelrohr, gef"allt mit Barytwasser, um die Bildung von kohlensaurem Baryt beobachten und diesen untersuchen zu k"onnen, und einem kleinen Rohr mit festem, feuchtem Kalihydrat. Das Steinpulver zeigte ein schwaches Glimmen und brannte sich rasch zimmtbraun, w"ahrend im Barytwasser ein starker Niederschlag entstand, der sich als kohlensaurer Baryt erwies. 1,680 Gramm Stein gaben 0,036 Kohlens"aure. Der erste Versuch der Art misslang, weil die gleichzeitige Bildung von schwefliger S"aure nicht vorausgesehen war. Aber bei beiden Versuchen erschien im Rohr jedesmal etwas Wasser, so sorgf"altig auch das Pulver zuvor getrocknet war, und zugleich ein wei"ser Rauch, der sich zu einem wei"sen, deutlich kristallinischen Sublimat verdichtete, das sich von einer Stelle zur andern sublimieren lie"s. Es war nicht zu erkennen, was es war. Es erschien auch, neben dem gebildeten Wasser, als eine andere kleine Menge des Steines in reinem Wasserstoffgas zum Gl"uhen erhitzt wurde. Da sich das Sublimat in einem Tropfen Alkohol l"oste und nach dessen Verdunstung wieder kristallinisch zur"uckblieb, so wurde mit dem letzten St"uckchen Stein noch der Versuch gemacht, die fl"uchtige Substanz durch sorgf"altig gereinigten hei"sen Alkohol auszuziehen. Nach dem Verdunsten hinterlie"s dieser dann, freilich nur in sehr kleiner Menge, eine farblose, weiche, nicht deutlich kristallinische Substanz, die sich beim Erhitzen an der Luft in unbestimmt riechenden wei"sen D"ampfen verfl"uchtigte, und die, in das Ende eines kleinen Rohres gebracht und erhitzt, schmolz, sich teilweise deutlich verkohlte, teilweise sich "olf"ormig an der Wand des Rohres hinaufzog, ohne nachher beim Erkalten zu erstarren. Als das Rohr dann an einer Stelle zum Gl"uhen erhitzt und der kleine Tropfen an die gl"uhende Stelle getrieben wurde, zersetzte sich die Substanz unter Abscheidung schwarzer Kohle, w"ahrend zugleich deutlich ein empyreumatischer Geruch zu bemerken war. Die zu diesen Versuchen angewandten Steinfragmente hatten ein zu frisches Ansehen und waren zu sorgf"altig aufbewahrt, als dass man diese Erscheinungen einer zuf"allig hineingekommenen Verunreinigung zuschreiben k"onnte.

Es w"urde zu den interessantesten und wichtigsten Betrachtungen f"uhren, wenn in einem Meteoriten das Vorkommen einer auf organische Materie deutenden Kohlenwasserstoff-Verbindung, mit der vielleicht auch der Kohlengehalt dieses Steines im Zusammenhang stehen k"onnte, mit Sicherheit nachzuweisen w"are. Schon Berzelius\footnote{\swabfamily{Poggendorffs Annalen. XXXIII. p. 114.}} fand bei der Analyse des erdigen Meteoriten von Alais in Frankreich eine kohlenhaltige Materie und ein braunes Sublimat, von dem er sagt: "`Dies ist ein mir g"anzlich unbekannter K"orper,"' und noch zu der Annahme geneigt, dass die Meteorsteine von einem anderen Weltk"orper herstammen, wirft er in Bezug auf die ungew"ohnliche Beschaffenheit jenes Steines von Alais die Frage auf: "`Enth"alt dieser erdige Stein wohl Humus oder Spuren von anderen organischen Verbindungen? Gibt dies m"oglicherweise einen Wink "uber die Gegenwart organischer Gebilde auf anderen Weltk"orpern?"' --- Mit dieser Vermutung, dass Meteoriten eine durch W"arme zersetzbare Verbindung enthalten k"onnten, steht das Feuerph"anomen bei dem Herabfallen und ihre geschmolzene Rinde in keinem Widerspruch, wenn man als sehr wahrscheinlich annimmt, dass diese K"orper nur ganz momentan einer au"serordentlich hohen Temperatur ausgesetzt gewesen sind, die nur die Oberfl"ache zu schmelzen, nicht aber die ganze Masse zu durchdringen vermochte.
\clearpage
\end{document}
