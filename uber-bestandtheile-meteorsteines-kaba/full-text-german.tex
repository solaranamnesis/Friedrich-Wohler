\documentclass[a4paper, 11pt, oneside]{article}
\usepackage[utf8]{inputenc}
\usepackage[T1]{fontenc}
\usepackage[ngerman]{babel}
\usepackage{fbb} %Derived from Cardo, provides a Bembo-like font family in otf and pfb format plus LaTeX font support files
\usepackage{booktabs}
\setlength{\emergencystretch}{15pt}
\usepackage{fancyhdr}
\usepackage{graphicx}
\graphicspath{ {./} }
\usepackage{microtype}
\usepackage[figurename=]{caption}
\begin{document}
\begin{titlepage} % Suppresses headers and footers on the title page
	\centering % Centre everything on the title page
	%\scshape % Use small caps for all text on the title page

	%------------------------------------------------
	%	Title
	%------------------------------------------------
	
	\rule{\textwidth}{1.6pt}\vspace*{-\baselineskip}\vspace*{2pt} % Thick horizontal rule
	\rule{\textwidth}{0.4pt} % Thin horizontal rule
	
	\vspace{1\baselineskip} % Whitespace above the title
	
	{\scshape\LARGE Sitzungsberichte\\[1.25pt] der Kaiserlichen\\[1.25pt] Akademie der Wissenschaften.\\[1.25pt] mathematisch-naturwissenschaftliche\\[1.25pt] klasse.\\[1.25pt]}
	
	\vspace{1\baselineskip} % Whitespace above the title

	\rule{\textwidth}{0.4pt}\vspace*{-\baselineskip}\vspace{3.2pt} % Thin horizontal rule
	\rule{\textwidth}{1.6pt} % Thick horizontal rule
	
	\vspace{1\baselineskip} % Whitespace after the title block
	
	%------------------------------------------------
	%	Subtitle
	%------------------------------------------------
	
	{\scshape Sitzung vom 11. November 1858.} % Subtitle or further description
	
	\vspace*{1\baselineskip} % Whitespace under the subtitle
	
    {\scshape\small Von dem c. M. Prof. F. Wöhler in Göttingen.} % Subtitle or further description
    
	%------------------------------------------------
	%	Editor(s)
	%------------------------------------------------
    \vspace*{\fill}

	\vspace{1\baselineskip}

	{\small\scshape Wien 1859.}
	
	{\small\scshape{Aus der K. K. Hof- und Staatsdruckerei.}}
	
	\vspace{0.5\baselineskip} % Whitespace after the title block

    \scshape Internet Archive Online Edition  % Publication year
	
	{\scshape\small Namensnennung Nicht-kommerziell Weitergabe unter gleichen Bedingungen 4.0 International} % Publisher
\end{titlepage}
\setlength{\parskip}{1mm plus1mm minus1mm}
\clearpage
\tableofcontents
\clearpage
\section{Sitzung vom 11. November 1858. --- Über die Bestandteile des Meteorsteines von Kaba in Ungarn.}
\paragraph{}
Die Fragmente von dem am 15. April 1857 bei Kaba in Ungarn gefallenen Meteoriten, die mir zur Analyse dienten, verdanke ich der Güte des Vorstandes des k. k. Hof-Mineralien-Kabinettes zu Wien, Herrn Dr. Hörnes, der über das Phänomen des Falles und die äußere Beschaffenheit des Steines im XXXI. Bande der Sitzungsberichte der mathematisch - naturwissenschaftlichen Classe der kaiserl. Akademie eine nähere Beschreibung mitgeteilt hat. Seinem Wunsche, die Analyse vorzunehmen, entsprach ich umso lieber, als das in der Tat ganz ungewöhnliche Aussehen dieses Steines auf eine ungewöhnliche Zusammensetzung schließen ließ.

Die mir übergebenen kleinen Fragmente waren ohne Rinde, hatten eine dunkelgraue Farbe und einen erdigen Bruch und waren leicht zerbrechlich und zerreiblich. In der erdigen grauen Grundmasse war hier und da ein weißes, und ein grünliches, ganz wie Olivin aussehendes Mineral zu bemerken. Die auch schon in mehreren anderen Meteorsteinen beobachteten sonderbaren leicht auslösbaren schwarzen Kügelchen waren in diesem Stein in ungewöhnlich großer Anzahl enthalten. Sie waren sehr spröde, zeigten nach dem Zerdrücken unter die Mikroskope im Innern einen leeren Raum und bestanden aus einem farblosen, sehr kristallinischen, und einem schwarzen Mineral. Die kleine, zu Gebote stehende Menge gestattete nicht, eine besondere Analyse davon zu machen. Von metallischen Teilchen war in diesen Fragmenten keine Spur zu entdecken; dennoch lenkten sie schwach die Magnetnadel ab, und aus dem Pulver ließen sich vermittelst des Magnetes sehr kleine Teilchen von metallischen Eisen ausziehen. Aus der oben erwähnten Beschreibung des ganzen Steines, wonach er auf der einen Seite viele glänzende Metallkörner enthält, ist daher zu schließen, dass er sehr ungleich gemengt sein muss. Das folgende analytische Resultat bezieht sich also nur auf den erdigen dunkelgrauen Theil dieses Steines. In 100 Theilen desselben wurden gefunden:
\begin{center}
    \begin{tabular}{ l r } 
    Kohle & 0,58.\\
    Eisen & 2,88.\\
    Nickel & 1,37.\\
    Kupfer & 0,01.\\
    Chromeisenstein & 0,89.\\
    Magnetkies & 3,55.\\
    Eisenoxydul & 26,20.\\
    Magnesia & 22,39.\\
    Thonerde & 5,38.\\
    Kalk & 0,66.\\
    Kalt (und Natron?) & 0,30.\\
    Manganoxydul & 0,05.\\
    Kieselsäure & 34,24.\\
    Kobalt & in unbestimmbarer Menge.\\
    Phosphor & in unbestimmbarer Menge.\\
    Unbekannte Materie & in unbestimmbarer Menge.\\
     & 98,50.\\
    \end{tabular}
\end{center}
\paragraph{}
Dieser Stein enthält also die gewöhnlichen Bestandteile der nicht metallischen Meteoriten, er ist ein Gemenge von einem durch Salzsäure leicht zersetzbaren Magnesia-Eisenoxydul-Silikat und von Silikaten, die durch diese Säure nicht zersetzt werden; er enthält außerdem kobalt- und phosphorhaltiges Nickeleisen, Schwefeleisen, Chromeisenstein und als ungewöhnlichen Bestandteil schwarze, amorphe Kohle. Was die unbekannte Materie betrifft, so will ich weiter unten noch einige Worte darüber sagen.

Es wurden zwei Analysen von den Steinen gemacht, die eine mit 2,827 Grm. durch Aufschließung mit kohlensaurem Kali-Natron, die andere mit 3,008 Grm. durch Flusssäure. Die erstere gab den obigen Kieselsäuregehalt, die andere, bei welcher der Verlust als Kieselsäuregehalt genommen werden musste, gab gerade 1 Prozent mehr.

Der Gehalt an metallischem Eisen konnte nicht direkt bestimmt werden, sondern wurde nach der Menge des Nickels berechnet, mit der Annahme, dass der Stein das den Meteoriten gewöhnliche, in Salzsäure schwer lösliche Nickeleisen enthalte. Denn er entwickelt mit dieser Säure kaum sichtbare Spuren von Wasserstoffgas. Die Säure löst aber selbst in der Kälte viel auf und diese Auflösung enthält dann viel Magnesia und Eisenoxydul. Es wurde daher der übrige Eisengehalt, mit Ausnahme des an Schwefel gebundenen, als Oxydul in Rechnung gebracht. Es wurden im Ganzen 34,57 Prozent Eisenoxyd erhalten, der gefundene Schwefelgehalt betrug 1,42 Prozent, entsprechend 3,55 Magnetkies. Dass der Stein diesen und nicht einfach Schwefeleisen enthalte, wird daraus wahrscheinlich, dass er mit Salzsäure erst in der Wärme deutlich Schwefelwasserstoff entwickelt und dass der Rückstand dann freien Schwefel enthält. - Kobalt und Phosphor waren mit Sicherheit nachzuweisen, ihre Mengen wären aber nur mit Anwendung von mehr Material zu bestimmen gewesen.

Der Kohlegehalt verrät sieh zunächst dadurch, dass der schwarze Rückstand von der Auflösung des Steines in Salzsäure selbst nach langem Kochen mit Königswasser schwarz blieb, dass er sich aber nach dem Auswüschen und Trocknen an der Luft rasch zimmtbraun brennen ließ, eine Eigenschaft, die auch der nicht mit Säure behandelte Stein hat. Zur quantitativen Bestimmung des Kohlenstoffes wurde eine abgewogene Menge des fein zerriebenen Steines in einem langsamen Strom von durch Kalihydrat sorgfältig gereinigtem Sauerstoffgas zum Glühen erhitzt, das Gas dann, zur Entfernung der gleichzeitig sich bildenden schwefligen Säure, durch ein Rohr mit Bleisuperoxyd und von da durch einen gewogenen Kaliapparat geleitet. Dieser bestand aus dem Liebig’schen Kugelrohr, gefällt mit Barytwasser, um die Bildung von kohlensaurem Baryt beobachten und diesen untersuchen zu können, und einem kleinen Rohr mit festem, feuchtem Kalihydrat. Das Steinpulver zeigte ein schwaches Glimmen und brannte sich rasch zimmtbraun, während im Barytwasser ein starker Niederschlag entstand, der sich als kohlensaurer Baryt erwies. 1,680 Gramm Stein gaben 0,036 Kohlensäure. Der erste Versuch der Art misslang, weil die gleichzeitige Bildung von schwefliger Säure nicht vorausgesehen war. Aber bei beiden Versuchen erschien im Rohr jedesmal etwas Wasser, so sorgfältig auch das Pulver zuvor getrocknet war, und zugleich ein weißer Rauch, der sich zu einem weißen, deutlich kristallinischen Sublimat verdichtete, das sich von einer Stelle zur andern sublimieren ließ. Es war nicht zu erkennen, was es war. Es erschien auch, neben dem gebildeten Wasser, als eine andere kleine Menge des Steines in reinem Wasserstoffgas zum Glühen erhitzt wurde. Da sich das Sublimat in einem Tropfen Alkohol löste und nach dessen Verdunstung wieder kristallinisch zurückblieb, so wurde mit dem letzten Stückchen Stein noch der Versuch gemacht, die flüchtige Substanz durch sorgfältig gereinigten heißen Alkohol auszuziehen. Nach dem Verdunsten hinterließ dieser dann, freilich nur in sehr kleiner Menge, eine farblose, weiche, nicht deutlich kristallinische Substanz, die sich beim Erhitzen an der Luft in unbestimmt riechenden weißen Dämpfen verflüchtigte, und die, in das Ende eines kleinen Rohres gebracht und erhitzt, schmolz, sich teilweise deutlich verkohlte, teilweise sich ölförmig an der Wand des Rohres hinaufzog, ohne nachher beim Erkalten zu erstarren. Als das Rohr dann an einer Stelle zum Glühen erhitzt und der kleine Tropfen an die glühende Stelle getrieben wurde, zersetzte sich die Substanz unter Abscheidung schwarzer Kohle, während zugleich deutlich ein empyreumatischer Geruch zu bemerken war. Die zu diesen Versuchen angewandten Steinfragmente hatten ein zu frisches Ansehen und waren zu sorgfältig aufbewahrt, als dass man diese Erscheinungen einer zufällig hineingekommenen Verunreinigung zuschreiben könnte.

Es würde zu den interessantesten und wichtigsten Betrachtungen führen, wenn in einem Meteoriten das Vorkommen einer auf organische Materie deutenden Kohlenwasserstoff-Verbindung, mit der vielleicht auch der Kohlengehalt dieses Steines im Zusammenhang stehen könnte, mit Sicherheit nachzuweisen wäre. Schon Berzelius\footnote{Poggendorffs Annalen. XXXII. p. 114.} fand bei der Analyse des erdigen Meteoriten von Alais in Frankreich eine kohlenhaltige Materie und ein braunes Sublimat, von dem er sagt: „Dies ist ein mir gänzlich unbekannter Körper“, und noch zu der Annahme geneigt, dass die Meteorsteine von einem anderen Weltkörper herstammen, wirft er in Bezug auf die ungewöhnliche Beschaffenheit jenes Steines von Alais die Frage auf: „Enthält dieser erdige Stein wohl Humus oder Spuren von anderen organischen Verbindungen? Gibt dies möglicherweise einen Wink über die Gegenwart organischer Gebilde auf anderen Weltkörpern?" - Mit dieser Vermutung, dass Meteoriten eine durch Wärme zersetzbare Verbindung enthalten könnten, steht das Feuerphänomen bei dem Herabfallen und ihre geschmolzene Rinde in keinem Widerspruch, wenn man als sehr wahrscheinlich annimmt, dass diese Körper nur ganz momentan einer außerordentlich hohen Temperatur ausgesetzt gewesen sind, die nur die Oberfläche zu schmelzen, nicht aber die ganze Masse zu durchdringen vermochte.
\clearpage
\end{document}
